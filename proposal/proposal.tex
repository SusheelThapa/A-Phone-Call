\documentclass[a4paper]{article}
\usepackage[utf8]{inputenc}
\usepackage{setspace}
\usepackage{hhline}
\usepackage[document]{ragged2e}
\usepackage{lipsum}

% Margins for document
\usepackage[left=1.5in, right=1.25in , top=1in, bottom=1in]{geometry}

% using package for TU logo(png)
\usepackage{graphicx}
\graphicspath{ {../resources/images/}}


\begin{document}

%% Front Cover %%
\newpage{

	% Latex code for Front Cover


	\begin{center}

		\includegraphics[width=4cm, height=4.3cm]{"proposal-tu-logo.png"} \\

		\vspace{5pt}

		\Large{
			\textbf{
				Tribhuvan University \\
				IOE, Pulchowk Campus \\
				Department of Electronics and Computer Engineering \\
			}

			\vspace{10pt}
		}
		% Three vertical lines

		\rule[-80pt]{1pt}{105pt}
		\hspace{20pt}
		\rule[-100pt]{1pt}{150pt}
		\hspace{20pt}
		\rule[-80pt]{1pt}{105pt}

		\vspace{20pt}
		OBJECT ORIENTED PROGRAMMING \\
		PROJECT PROPOSAL\\
		For \\

		\textbf{
			A PHONE CALL \\
			\vspace{20pt}
		}
		\vspace{85pt}




		\textbf{\underline{SUBMITTED BY:}}
		\hfill
		\textbf{\underline{SUBMITTED TO:}} \\

		\vspace{15pt}
		\begin{tabular}{lrr}
			\Large	Susheel Thapa [077BCT092]     & DEPARTMENT OF       \\
			\Large	Saurav Kr. Mahato [077BCT079] & ELECTRONIC COMPUTER \\
			                                     & ENGINEERING         \\
		\end{tabular}
	\end{center}

}

%% Acknowledgement %%
\newpage{

	% Latex Code for Acknowledgement

	\setlength{\parindent}{0pt}
	\begin{center}
		\textbf{ {\section*{\huge {Acknowledgement}}}}
	\end{center}

	%%vertical spacing
	\vspace{30pt}

	\Large{

		\justify First of all we would like to thank to our OOP instructor Daya Sagar Baral , Lokh Nath Regmi and Shanti Tamang we helped us
		to learn OOP in C++. Along with learning they provided us the opportunity to apply the concept of OOP to manage our project and use its feacture like  encapsulation, abstraction, inheritance and polymorphism.

		\justify \noindent Would also like to thank whole team of Deapartment of Electronics and Computer Engineering to provide us the
		opportunity to expose our skills and express our learning throught projects.

		\justify \noindent Thanks to all our seniors as well who directly or indirectly help us through the learning process and made our journey easy through different session and workshops.

		\justify \noindent Finally, thanks to all our friends and classmates who helped in discussing different idea and workflow of the project along with providing good suggestions.

	}
}

%% Table of contents %%
\newpage
{

	% Latex code for Table of contents

	\begin{center}
		\textbf{ \huge{Table of contents}}
		\vspace{30pt}
	\end{center}


	\vspace{20pt}

	\Large{

		\textbf{1. Introduction	.........................................................** \\ }
		\vspace{20pt}
		\textbf{2. Objectives  ............................................................**\\}
		\vspace{20pt}
		\textbf{3. Existing System ...................................................**\\}
		\vspace{20pt}
		\textbf{4. Proposed System .................................................**\\}
		\vspace{20pt}
		\hspace{15pt}\textbf{4.1 Introduction ....................................................** \\}
		\vspace{20pt}
		\hspace{15pt}\textbf{4.2 System Block Diagram ....................................** \\}
		\vspace{20pt}
		\textbf{5. Methodology ........................................................**\\}
		\vspace{20pt}
		\hspace{15pt}\textbf{5.1 Tool we will used .............................................** \\}
		\vspace{20pt}
		\hspace{15pt}\textbf{5.2 Approach to the project ..................................**  \\}
		\vspace{20pt}
		\textbf{6. Project Scope .......................................................** \\}
		\vspace{20pt}
		\textbf{7. Project Schedule ..................................................**}
		\vspace{20pt}

	}

}



%% Introduction %%
\newpage{

	% Latex Code for Introduction

	\begin{center}

		\textbf{ \huge{1.\hspace{25pt}Introduction}}

	\end{center}

}

%% Objectives %%
\newpage{

	% Latex Code for Objectives

	\begin{center}

		\textbf{ \huge{2.\hspace{25pt}Objectives}}

	\end{center}

}

%% Existing System %%
\newpage{

	% Latex Code for Existing System

	\begin{center}

		\textbf{ \huge{3.\hspace{25pt}Existing System}}

	\end{center}

}

%% Proposed System %%
\newpage{

	% Latex Code for Proposed System

	\setlength{\parindent}{0pt}
	\begin{center}
		{\section*{\huge {\textbf{4.\hspace{25pt}Proposed System}}}}
	\end{center}

	\vspace{30pt}

	\Large{
	\subsection*{\Large{4.1 Introduction}}

	\justify *****************************************
	\justify *****************************************
	\justify *****************************************\\
	% \vspace{20pt}
	\setlength{\parindent}{0pt}
	\subsection*{\Large{4.2 System Block Diagram}}
	}

}

%% Methodology %%
\newpage{

	% Latex Code for Methodology

	\begin{center}

		\textbf{ \huge{5.\hspace{25pt}Methodology}}

	\end{center}

}

%% Project Scope %%
\newpage{

	% Latex Code for Project Scope

	\setlength{\parindent}{0pt}
	\begin{center}
		\textbf{ {\section*{6.\hspace{25pt}\huge {Project Scope}}}}
	\end{center}

	\vspace{30pt}
	\setlength{\parindent}{0pt}
	\Large
	{
		\justify Our project (A Phone Call) was initiated with the aim to demostrate the client-server connection like real world's telecommunication.
		The key objectives of this project is to make client-server model and show the interaction of different client through the server.

		\justify This project easily shows the real life telecommunication. It shows how we call different person in real life and what is going behind the scene.
		Our milestone is to acheive that simple prototype of the telecommunication.

		\justify Two different database is used in client and sever for purpose of
		storing different information. SDL as a graphic is used to match the dialpad model like we see in real life phone.

	}
}

%% Project Schedule %%
\newpage{

% Latex Code for Project Schedule

\setlength{\parindent}{0pt}
\begin{center}
	\textbf{ {\section*{7.\hspace{25pt}\huge {Project Schedule}}}}
\end{center}

\vspace{30pt}
\setlength{\parindent}{0pt}
\Large
{
	\justify The whole project can be divided into two part client side and sever side. Client side will take more time as it will contain the graphic part or visible part.
	As discussed in the methodology, the project will all together take around 20 to 30 days.

	\justify All the phases in making this project and their time requirement is given below in table: \\

	\begin{center}
		\begin{tabular}{|c|c|}
			\hline
			Phases:            & Days Required:     \\
			\hline
			****************** & ****************** \\
			****************** & ****************** \\
			****************** & ****************** \\
			****************** & ****************** \\
			****************** & ****************** \\

			\hline
		\end{tabular}

	\end{center}


}

}
\end{document}
